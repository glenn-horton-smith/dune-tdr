\begin{table}[htp]
  \caption{Specification for TOP-15 \fixmehl{ref \texttt{tab:spec:lar-n-contamination}}}
  \centering
  \begin{tabular}{p{0.2\textwidth}p{0.75\textwidth}} 
     \rowcolor{dunesky}
    \newtag{TOP-15}{ spec:lar-n-contamination } 
                & Name: LAr nitrogen contamination    \\ 
    Description & The nitrogen contamination in the LAr shall remain below 25 ppm in order not to significantly affect the number of photons that reach the detectors (for both fast and late light components).   \\  \colhline
    
    Specification &  < \num{25} ppm \\   \colhline
    
    Rationale &  { Production quenching  prevents most late light from being collected at levels of 10ppM and above.  The attenuation length in liquid argon with 50 ppM of Nitrogen contamination is roughly 3m, which starts to signifcantly affect the number of photons that reach the detectors (for both fast and late light components). } \\ \colhline
    Validation &{ Nitrogen contamination levels can be measured both from gas analyzers and from studies of the effective shortening of the slow component of generated scintillation light orginating from nitrogen quenching.  Comparison of the two will inform light simulation models. } \\    
   \colhline
  \end{tabular}
  \label{tab:spec:lar-n-contamination}
\end{table}