\begin{table}[htp]
  \caption{Specification for SP-FD-13 \fixmehl{ref \texttt{tab:spec:fe-peak-time}}}
  \centering
  \begin{tabular}{p{0.2\textwidth}p{0.75\textwidth}} 
     \rowcolor{dunesky}
    \newtag{SP-FD-13}{ spec:fe-peak-time } 
                & Name: Front-end peaking time    \\ 
    Description & The FE peaking time shall be set so as to optimize vertex resolution.    \\  \colhline
    Specification (Goal) &  \SI{1}{\micro\second}  ( Adjustable so as to see saturation in less than \SI{10}{\%} of beam-produced events ) \\   \colhline
    
    Rationale &   Driven by the need for optimal vertex resolution, which is itself determined from single-track resolution and the ability of the reconstruction algorithms to separate two nearby tracks.  Value of 1 us is based on 5mm spacing between anode planes. Reconstruction performance depends on anode wire spacing, anode-to-cathode spacing, bias voltages, and distance between anode planes (as well as electron drift lifetime).  \\ \colhline
    Validation & ProtoDUNE vertex reconstruction in multi-track events will demonstrate the validity of the requirement. Simulation studies could illuminate the trade-off between shaping time and noise if the new electronics is able to achieve lower noise at longer shaping times.  \\
   \colhline
  \end{tabular}
  \label{tab:spec:fe-peak-time}
\end{table}