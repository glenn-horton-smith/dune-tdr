\section{Production and Assembly}
\label{sec:fdsp-tpcelec-production}

In this Section we discuss the production and assembly plans,
including the plans for the spares required during the detector
construction and for operations, and then the plans for
procurement, assembly, and finally quality control.

%%%%%%%%%%%%%%%%%%%%%%%%%%%%%%%%%%%

\subsection{Spares Plan}
\label{sec:fdsp-tpcelec-production-spares}

The \dword{apa} consortium plans on building 152 \dword{apa}s
for the first \dword{sp} detector. This means that at least
3,040 \dwords{femb} with the corresponding bundles of cold
cables will be required for the integration (3,040 power cables, 3,040 data cables,
and 1,216 bias voltage cables; half the cables will be long enough for 
integration on the top \dword{apa}s, while the other half will
be compatible with the bottom \dword{apa}s). To have spare \dwords{femb}, the \dword{ce} consortium plans to
build at least 3,200 \dwords{femb}, 5\% more than necessary. If more spares are needed during the
\dword{qc} process or during integration, additional 
\dwords{femb} can be produced quickly if there are sufficient components on hand
that require long lead times. For components with long lead times, we plan to keep on hand a
larger number of spares. The \dwords{asic} require a long lead time; a plan for those spares is
discussed below. For other discrete components
(capacitors, resistors, connectors, voltage regulators, oscillators),
plans will be put in place once the final design of the \dword{femb}
is available and vendors contacted. %Please note that the
%size of the \dword{femb} production is such that the total number
%of some components exceeds the usual stock of some distributors.

For the \dwords{asic}, the number of spare chips is driven by the fact that fabrication
requires batches of 25 wafers at a time. Given the expected final size of each design, the
expected number of chips per wafer is about 700 for \dword{larasic}, 930 for \dword{coldadc},
230 for \dword{coldata}, and 220 per wafer for \dword{cryo}. These numbers are based on the
assumption that \dword{coldadc} and \dword{coldata} are fabricated on the same wafer. To
estimate the number of usable chips for installation on the \dwords{femb}, we assume that
10\% of the chips will fail during the \dword{qc} process described later in this Section,
and an additional 5\% of the chips will fail during dicing and packaging. With these
assumptions one would need at least 43 \dword{larasic} wafers, 33 \dword{coldadc} wafers,
33 \dword{coldata} wafers, and 35 \dword{cryo} wafers for one \dword{sp} detector. Wafers
must be ordered in batches of 25 which implies that we will have a significant number of
spares. The number of spare chips available can be reduced if wafers are purchased for two
\dword{sp} detectors at a time, however, the wafers are relatively inexpensive and the chosen
processes may not be available after a few years so generous spares of these custom devices
are likely to be advisable. 

In general, for other components, we plan to procure between 5 and
10\% additional components for spares for the construction of the first \dword{sp}
detector. We will need more spares for components that have
a larger risk of damage during integration and 
installation. For example, for cold cables, we
plan for 10\% extra for spares for the cables on the bottom \dword{apa} because
they must be routed through the \dword{apa} frames, but
for the top \dword{apa}, we foresee needing only 5\% extra for spares.
Assuming we will have unused spares from the first detector, we will reduce the number of
spares for the second \dword{sp} detector.

The components on top of the cryostat (power supplies, bias
voltage supplies, cables, \dwords{wiec} with their \dword{wib}
and \dword{ptc} boards) can be replaced while the
detector is in operation. For these components, additional spares may be required
during the \dunelifetime operation period of the \dword{dune} detector.
The initial plan is to purchase 10\% additional components for spares for the first
\dword{sp} detector and use them for a second detector as well
(i.e., effectively having 5\% additional components for spares). Once the design of
the \dwords{wib} is finalized, we will decide if 
extra spares should be purchased for \dwords{fpga} and optical
transmitters and receivers. These are commercial components 
that may no longer be available after a certain number of 
years of operation, which could prevent the \dword{ce} consortium from
fabricating additional spare \dwords{wib} if required. This
risk is discussed in Section~\ref{sec:fdsp-tpcelec-risks-commissioning}, one that
could be alleviated by placing commercial components
on mezzanine cards to minimize any necessary redesign of
boards if these components are no longer available.
We can also stock additional components
if market trends show that the components will  
become harder or impossible to find.

%%%%%%%%%%%%%%%%%%%%%%%%%%%%%%%%%%%
\subsection{Procurement of Parts}
\label{sec:fdsp-tpcelec-production-procurement}

Constructing the detector components for \dword{dune}, a responsibility 
of the \dword{ce} consortium, requires many large procurements that 
must be carefully planned to avoid delays. For the \dwords{asic}, the 
choice of vendor(s) is made at the time the technology used in designing 
the chips is chosen. For almost all other components, several vendors 
will bid on the same package. Depending on the requirements of the funding
agency and of the responsible institution, this may require a lengthy
selection process. The cold cables used to transmit data from the
\dwords{femb} to the \dwords{wib} represent a critical case. In this case
a technical qualification, including tests of the entire cold chain (from the \dword{femb}
to the receiver on the \dword{wib}), is required. Another problem is the 
large numbers of components required. In some cases, the number of components 
of a given type (resistors, capacitors) may far exceed the number of components
that the usual resellers keep in stock. This will 
require careful planning to avoid stopping the assembly chain for
the \dwords{femb}, for example, because 
one kind of component runs short.

%%%%%%%%%%%%%%%%%%%%%%%%%%%%%%%%%%%
\subsection{Assembly}
\label{sec:fdsp-tpcelec-production-assembly}

The \dword{ce} consortium plans to minimize
the amount of assembly work at any one of the participating
institutions. When assembly work is required, it will be performed
by external companies; examples are the installation of surface 
mount components, \dwords{asic}, \dwords{fpga} on the printed 
circuit boards for the \dwords{femb} and the \dwords {wib}, or
the assembly of the crossing tube cable supports. One of the few
exceptions is the assembly of the \dwords{wiec} that involves
mechanical and electrical connections of the backplane and of
the supports of the crates. Another component that possibly 
will be assembled at one of the consortium institutions is
the plug attached to the cold cables, which is used to protect 
the \dwords{femb} from \dword{esd} damage. During the engineering
phase and for components fabricated in small quantities, 
like boards used for testing other components, the plan is to
have one of the consortium's institutions assemble the components.
Further assembly of components and integration with parts
provided by other consortia members takes place at the \dword{itf} and
at \dword{surf}. This is discussed in Section~\ref{sec:fdsp-tpcelec-integration}.

%%%%%%%%%%%%%%%%%%%%%%%%%%%%%%%%%%%
\subsection{Quality Control}
\label{sec:fdsp-tpcelec-production-qc}

Once the \dword{apa}s are installed inside the cryostat, only
limited access to the detector components will be available to the \dword{ce}
consortium. After the \dword{tco} is closed, no access to detector
components will be available; therefore, they should be constructed to
last the entire lifetime of the experiment (\dunelifetime). This
puts very stringent requirements on the reliability of these
components, which has been already addressed in part through 
the \dword{qa} program discussed in Section~\ref{sec:fdsp-tpcelec-qa}. The
next step is to carefully apply stringent \dword{qc} procedures for  
detector parts to be installed in the detector.
All detector components installed inside the cryostat will
be tested and sorted before they are prepared for integration
with other detector components and installed. The full
details of the \dword{qc} plan have not been put in place
yet, and the specific selection criteria for the components will
be defined only after the current design and
prototyping phase are completed. For each detector component, a preliminary
version of the \dword{qc} program will be developed before the engineering design
review. The program will then be used for
qualification of components fabricated during 
pre-production. It will be modified as needed before the production
readiness review that triggers the start of production of detector components
used for assembling the detector.

Some of the requirements for the \dword{qc} plans can
be laid out now based on the lessons learned
from constructing and commissioning the \dword{pdsp}
detector. Experience with \dword{pdsp} shows that a small fraction
($\approx4$\%) of the \dword{fe} \dwords{asic} that pass the
qualification criteria at room temperature fail the tests
when immersed in \lntwo. Therefore, we plan to test all \dwords{asic} in \lntwo
before they are mounted on the \dwords{femb}; 
cryogenic testing of the \dwords{femb} is also planned. The goal of testing 
the \dwords{asic} in \lntwo is to minimize the need
to rework the \dwords{femb}. This is more important if 
the three \dword{asic} solution is chosen for the
\dword{femb}. In this case, there are 18 \dwords{asic} on the
\dword{femb}, and an upper limit of 2\% on the fraction of
\dwords{femb} that require reworking, translates into a requirement 
that less than 0.11\% of the \dwords{asic} fail during immersion in \lntwo.
If the \dword{cryo} solution is chosen for the \dwords{asic}
to be used on the \dwords{femb}, the 2\% requirement for the
number of \dwords{femb} to be reworked translates to a 
maximum failure rate of 1\%, given that only two
\dwords{asic} are on the \dwords{femb}. Based on the \dwords{pdsp}
experience, discrete components like resistors and capacitors
need not undergo cryogenic testing before they are installed
on the \dwords{femb}. Capacitors and resistors are commonly
sold in reels of a few thousand components, which
should be enough, usually, to fabricate ten
\dwords{femb}. For these components we are planning to
perform cryogenic tests on samples of a few components
from each reel, prior to using the reel in the assembly of
\dwords{femb}. Some other components installed on
the \dwords{femb}, like voltage regulators and crystals, most probably will have to be qualified like the \dwords{asic}
in \lntwo before being mounted on the \dwords{femb}.

For the large numbers of
\dwords{asic} required for one \dword{dune} \dword{sp} detector
(6,000 or 54,000 chips depending on the \dword{asic} solution
chosen), manual testing of the chips requires excessive 
resources and, based on the lessons learned from  
constructing  \dword{pdsp}, would lead to unacceptable rejection
factors. Ideally, the entire testing
process would be performed using a robotic system, where 
a robotic arm picks up the \dword{asic} from a tray, places
it on a test board, and holds it in place while the test
is performed, and then sorts it into a second tray
depending on the test result. The requirement that the test
be performed in \lntwo prevents us from using this
scheme. 

One of the biggest problems observed while constructing
\dword{pdsp} is placing the chips in
the sockets, since the issues with condensation
have been addressed with the development of the \dword{cts}.
To overcome problems with manually placing chips into
the sockets, we plan to develop a robotic system
to perform this operation. Once the \dwords{asic} are 
placed on test boards, they will be moved manually into 
upgraded versions of the current \dword{cts} that can
house multiple test boards. At the end of the testing
procedure, the robotic system will then remove
the chips from the test boards and sort them according to
the test results. Based on the experience with the tests of
the \dword{pdsp} \dwords{asic}, as well as from other experiments,
we plan to have the sockets on the 
test boards cleaned and to replace them after a certain number of
testing cycles. To test all the \dwords{asic} required for
constructing one \dword{sp} detector, we plan on having two to six test sites, each equipped with a
robotic system and an upgraded \dword{cts}. All tests
will be performed following a common set of instructions
at all sites. To ensure that all sites produce similar
results, we will have a reference set of \dwords{asic}
that will be initially used to cross-calibrate the 
test procedures among sites and then to check the 
stability of the test equipment at each site. Test results will
be stored in a database, and criteria will be developed
for the acceptance of \dwords{asic}. The acceptance rate 
will be monitored, and in case of problems,
the failures will be analyzed and root cause analyses
will be performed. If necessary, the test program will be stopped
at all sites while issues are investigated. In the case of \dword{cryo},
since each \dword{femb} will only have two chips, it may be possible to bypass
chip-level testing altogether. If the chip-level failure rate is low enough,
it may be sufficient to simply test assembled \dwords{femb} and reject or rework
those that fail the tests.

Before assembly, the printed circuit boards for the
\dwords{femb} will be tested by the vendor for electrical
continuity and shorts. The usual approach for particle physics
experiments is to perform a visual inspection of the boards
before installing the discrete components and 
the \dwords{asic}. This inspection will be repeated after 
installation and before the functionality test, which for \dword{dune} will
be performed in \lntwo. The specifications on vias and pads for the printed
circuit boards for the \dwords{femb} are not at the edge of the
industrial vendor capabilities, and therefore, we do not
expect these inspections to be absolutely necessary. We will
perform visual inspections on a sample of the 
production, but we will also investigate the possibility
of using other, possibly automatic, inspection methods for
the bulk of the production. After assembly, 
each \dword{femb} will be tested in \lntwo using
the current \dword{cts}. Nine \dwords{cts} have already been
fabricated and are being distributed among the institutions
in the consortium. 

The test procedures are likely to be
very similar to the ones adopted for \dword{pdsp}, with the main
difference that the tests will not be performed with
the final cables to be used in the experiment but 
with a set of temporary cables. The final cables will be tested
separately as described below. The tests of the \dwords{femb}
are performed using the \dword{cts}, which allows a turnaround
time of about one hour per \dword{femb}. In the
test, the \dword{femb} is connected to a capacitive load that
simulates the presence of \dword{apa} wires. This allows
connectivity checks for each channel as well as measurements of
the baseline and of the \rms of the noise. Calibration 
pulses will be injected in the front-end amplifier, digitized,
and read out. These injected pulses will also be used
to determine the calibration constants of the \dword{adc}. 
The test set up requires one \dword{wib} and
a printed-circuit board similar to those used on the cryostat
penetration, allowing simultaneous testing of four \dwords{femb}.
A standalone 12V power supply is required, and the readout
of the \dword{wib} uses a direct Gb ethernet connection to
a PC. The set up used for \dword{asic} testing is similar.
In both cases, the data can be processed locally on the PC,
and the results from the tests and calibrations are then stored 
in a database. The plan is to have the capability to retrieve  
these test and calibration results throughout the entire life
of the experiment. As in the case of \dwords{asic} testing,
we will monitor the test results to ensure that all
sites have similar test capabilities and yields and to
identify possible problems during production.
Further tests will be performed on the \dwords{femb}
before and after their installation on the \dword{apa}s as
discussed in Section~\ref{sec:fdsp-tpcelec-integration-timeline}.

The final component provided by the \dword{ce} consortium
and installed inside the cryostat is the ensemble of cold
cables: the cables carrying the bias voltage for the \dword{apa}
wires; the electron diverters and the field cage termination electrodes;
the cables carrying the low voltage power to the \dwords{femb};
and the data cables that carry the clock and control signals
to the \dwords{femb} and are also used for the readout. It is neither
feasible nor necessary to test these cables in \lntwo
because they will usually perform better at cold temperatures than 
room temperatures. We will perform checks on all cables 
during production, before they are installed and 
connected to the \dwords{femb}. A further test will take place
when the \dword{apa}s are tested in the cold boxes at the \dword{itf}
and at \dword{surf}. The checks performed at room temperature include
measuring the resistance of all the cables, checking for
shorts, and for the data cables, measuring the
eye diagram when transmitting data at the same rate as from the
\dword{femb}. Connectors will be visually inspected 
to ensure they show no sign of damage.

Stringent requirements must be applied to the cryostat
penetrations, to avoid argon leaks. The cryostat penetrations 
have two parts: the first is the crossing tube with its spool pieces,
and the second one is the three flanges used for
connecting the power, control, and readout electronics with the
\dword{ce} and the \dword{pds} components inside the
cryostat. On each cryostat penetration there are two flanges for
the \dword{ce} and one for the \dword{pds}. The crossing
tubes with their spool pieces are fabricated by industry and tested
by vendors to be leak and pressure proof. The flanges are assembled
in consortia institutions responsible for the \dword{ce} and \dword{pds}; the
flanges must undergo both electrical and mechanical tests to ensure their
functionality. Electrical tests comprise checking all the
signals and voltages to ensure they are passed properly between the two sides of the
flange and that there are no shorts. Mechanical tests involve 
checking that the flange itself is leak and pressure proof. Further
leak tests are performed after the cryostat penetrations are installed
on the cryostat and later after the \dword{ce} and \dword{pds}
cables are attached to the flanges. These leak tests are
performed by releasing helium gas in the cryostat penetration and
checking for the presence of helium on top of the cryostat. Similar
tests were performed during the \dword{pdsp} installation.

All other detector components that are a responsibility of
the \dword{ce} consortium can be replaced, if necessary,
even while the detector is in operation. Still, every component
will be tested before it is installed in \dword{surf} to ensure a
smooth commissioning of the detector. The \dwords{wiec} will be
assembled and tested with all the \dwords{wib} and \dword{ptc}
installed. Testing requires a slice of the \dword{daq} back-end,
power supplies, and at least one \dword{femb} to check all 
connections. All cables between the bias voltage supplies and
the end flange and all the cables between the low voltage power
supplies and the \dwords{ptc} will be tested for electrical
continuity and for shorts. All power supplies will undergo a
period of burn-in with appropriate loads before being installed
in the cavern. Optical fibers will be tested by measuring the
eye diagram for data transmission at the required speed. All
test equipment used for qualifying the components to be installed
in the detector will be either transported to \dword{surf} or duplicated
at \dword{surf} to be used as diagnostic tools during operations.

%%%%%%%%%%%%%%%%%%%%%%%%%%%%%%%%%%%
\subsection{Test Facilities}
\label{sec:fdsp-tpcelec-production-facilities}

The \dword{qc} plan described in the previous Section requires
multiple test stands that must be put in place and used 
before the beginning of production. For the \dwords{asic}
testing, a set up similar the one for \dword{cts} (see
Section~\ref{sec:fdsp-tpcelec-qa-initial}) will
be used. In this set up, several test boards housing up to 24
chips will be immersed in \lntwo before running the
electronics tests; the test boards will later be warmed to room temperature in 
a nitrogen atmosphere to avoid condensation on the chips and
boards. As mentioned, placing the chips on the test 
boards will be performed using a robotic system. The test set ups,
one for each kind of \dword{asic}, will be an evolution of those
used initially for characterizing the \dword{asic} and similar
to the set ups used for qualifying the chips used in the \dword{pdsp}
construction. The tests of the \dwords{femb} will be performed with
set ups that include using \dwords{cts} for the cryogenic part
but are otherwise simple evolutions (with newer \dwords{wib})
of the set ups used to characterize the \dwords{femb}
for \dword{pdsp}. Cold and warm power and bias voltage cables will
be characterized with test stations with the relevant 
power supplies and some cable testing equipment; these cables will most
likely be \dword{cots} components.
For the test of the data cables, we will probably rely on a set up
using waveform generators and an high end oscilloscope that 
can handle 2.56 Gbps signals and measure eye diagrams. 
Burn-in stations, with custom designed loads, may be required for 
the commercial low voltage power and bias voltage supplies.
A test set up to check \dwords{wiec} with their \dwords{wib}
and \dwords{ptc} will require a minimal \dword{daq} back-end that the
\dword{daq} consortium should provide.

Given the delay between the beginning of production of 
\dword{apa}s and production of the components that are a responsibility of
the \dword{ce} consortium, as well as the availability of the
integration set up at \dword{surf} in front of the cryostat, it is
desirable to  integrate the \dwords{femb}
on some of the \dword{apa}s and perform tests in cold
boxes. During the pre-production phase, this can be done at CERN,
but later, the integration and the
tests should be done at the \dword{itf}, which then would require a cold box,
as well as a full power, control, and readout system, similar
to the one described in Section~\ref{sec:fdsp-tpcelec-integration-qc}.
