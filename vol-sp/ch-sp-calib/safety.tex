%\section{Safety}
%\label{sec:sp-calib-safe}

%We consider here only personal risk to humans, that apply in the prototyping phase, including ProtoDUNE deployment, and also during integration and commissioning at the DUNE far detector site. Risks of damaging the systems and/or other DUNE detector components are discussed in Section~\ref{sec:sp-calib-risks}.

This section discusses risks involving personnel safety. Detector safety and risks involving damage to detector components are discussed in Section~\ref{sec:sp-calib-risks}.

Human safety is of critical importance during all phases of the calibration work, including R\&D, laboratory testing, prototyping phase (including ProtoDUNE deployment), and also during integration and commissioning at the DUNE far detector site. Safety experts review and approve the initial safety planning for all phases of work as part of the initial design review, as well as before implementation. All documentation of component cleaning, assembly, testing, and installation will include a section on relevant safety concerns and will be reviewed during appropriate pre-production reviews.

Several areas are of particular importance to calibration are:
\begin{itemize}
\item {\bf Underground lab safety:} All personnel working underground (UG) or in other installation facilities, must follow all appropriate safety training, and be provided with the required personal protective equipment. Risks associated with the installation and operation work of the calibration device include e.g.: working at heights, confined space access, falling objects overhead operations, electrical safety. Appropriate safety procedures including lift and harness training will be designed and reviewed for working at heights. For falling objects, the corresponding safety procedures, including proper helmet use and a well restricted safety area, will be included in the safety plan.

\item {\bf Eye safety:} The laser system requires the operation of a class 4 laser. This requires an interlock on the laser box enclosure, and only trained personnel present in the cavern for the one-time alignment of the laser upon installation in the feedthroughs.

\item {\bf Radiation:} The gammas from neutron capture on hydrogen could bring a potential radiation safety concern for the PNS. The design of key safety systems (custom shielding and moderator) for the PNS will be discussed with safety experts at member institutions
%at CERN and at MSU 
prior to operation at ProtoDUNE. In particular, the entire system will be assembled in a neutron shielded room and tested to confirm there is no leak of neutrons. The system will also have a neutron monitor which can be used to provide an interlock. The Radioactive source system also poses a radiation risk, which will be mitigated with a purge-box for handling, and a shielded storage box and area with lockout-tagout procedure, also applied to the gate-valve on top of the cryostat.

\item {\bf High voltage safety:} Some of the calibration devices will use high voltage. Fabrication and testing plans will show compliance with local \dword{hv} safety requirements at any institution or laboratory that conducts testing or operation, and this compliance will be reviewed as part of the standard review process.

\item {\bf Hazardous chemicals:} Hazardous chemicals (e.g., epoxy compounds used to attach components of the system) and cleaning compounds used will be documented at the consortium management level, with a MSDS (material safety data sheet) and approved handling and disposal plans in place.

\item {\bf Liquid and gaseous cryogens:} cryogens will most likely be used in testing (e.g. LN and \lar) of calibration devices. Full hazard analysis plans will be in place at the consortium management level for full module or module component testing that involves cryogens. These safety plans will be reviewed in appropriate pre-production and production reviews.

\end{itemize}

%We consider risks to the calibration systems themselves, and also to other DUNE materials or systems. \fixme{This may be a shared concern. We want to avoid bumping/breaking components as they are checked, installed and commissioned in DUNE. Special care will need to be taken to install components and do checks stepwise.} 
%• Other equipment (DSS, HV, APA) hitting laser periscopes, if already installed inside cryostat


%{\bf Damage to the photon detection system by the laser:}  To mitigate possible damage to the PD system, software will be used to block the beam while the mirrors are stopped or when laser light is directed at the PD system. Initial discussion with PDS indicates that this may not be a significant issue. \fixme{relationship between this and interface with PD?}

%{\bf Radiation damage to DUNE components:} The activation caused by the PNS is being studied and will be known by ProtoDUNE testing for the PNS at neutron flux intensities and durations well above the run plan. \fixme{May also need to reference background TF. Add RS system.}

%\todo{We have started discussions about electrical safety and grounding, and will update this once formal documents are prepared for that.} 
%Slides: https://indico.fnal.gov/event/16764/session/8/contribution/39/material/slides/2.pdf