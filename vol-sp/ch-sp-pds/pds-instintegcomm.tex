%%%%%%%%%%%%%%%%%%%%%%%%%%%%%%%%%%%%%%%%%%%%%%%%%%%%%%%%%%%%%%%
\section{Installation, Integration and Commissioning}
\label{sec:fdsp-pd-install}
%\metainfo{Content: Onel, Kemp, Warner}

%>> Revision: Ernesto Kemp, Yasar Onel, David Warner Nov/23/2018 >>>>>>>>>>>>>

%=================================================
\subsection{Transport and Handling}
\label{sec:fdsp-pd-install-transport}


Following assembly and testing of the \dword{pd} modules they will be packaged and shipped to the \dword{fd} site for checkout and any final testing prior to installation into the cryostat. Handling and shipping procedures will depend on the final environmental requirements determined for the photon detectors, and will be specified prior to the \dword{tdr}.

A testing plan is developed to determine environmental requirements for photon detector handling and shipping. The environmental conditions apply for both surface and underground transport, storage and handling. Requirements for light (UV filtered areas), temperature, and humidity exposure are also developed.

Handling procedures that ensure environmental requirements are developed. This will include handling at all stages of component and system production and assembly, testing, shipping, and storage. It is likely that \dword{pd} modules and components will be stored for periods of time during production and prior to installation into the \dword{fd} cryostats. Appropriate storage facilities need to be constructed at locations where storage will take place. Shipping and storage containers need to be designed and produced. Given the large number of photon detector modules to be installed in the \dword{fd}, it will be cost effective to take advantage of reusable shipping containers.

Documentation and tracking of all components and \dword{pd} modules will be required during the full production and installation schedule. Well defined procedures are in place to ensure that all components/modules are tested and examined prior to, and after, shipping. Information coming from such testing and examinations will be stored in a hardware database.

%  DWW>  Probably part of the integration section?  An Integration and Test Facility (ITF) will be constructed at a location to be decided by the collaboration/project for the integration of the \dword{pd}s into \dwords{apa}. Transportation to and from ITF should be carefully planned. The \dword{pd}S units will be shipped from the production area in quantities compatible with the \dword{apa} transport rates.
    
%Operations: The \dword{pd}S deliveries will be stored in temperature and humidity controlled storage area. Their mechanical status will be inspected.

%Transportation to SURF: The delivery to SURF will be such that the storage time before integration will be at most two weeks.


%%%%%%%%%%%%%%%%%%%%%%%%%%%%%%%%%%%
\subsection{Integration with APA and Installation}
\label{sec:fdsp-pd-install-pd-apa}

Integration of the \dword{pd} modules into the \dword{apa} frame will happen at the \dword{itf}. \dword{apa}s will be oriented horizontally for \dword{pd} modules integration. Experts from both groups will work with the installation team.  
An electrical test with \dword{apa}/\dword{pd}S/\dword{ce} will be performed 
%at the integration facility 
in the underground in a cold box, after the integration of \dword{pd}S and \dword{ce} on the \dword{apa} frame has been completed. During the cold test \dword{pd} modules will be operated for dark count measurements and LED illumination.

The \dword{apa} consortium will be responsible for the transportation of the integrated \dword{apa} frames from the integration facility to the LBNF/SURF facility. 
The \dword{uit} team, under supervision of the \dword{apa} group, will be responsible to move the equipment into the clean room. 
Work on the 2-\dword{apa} connection and inspection underground, prior to installation in the cryostat, is performed by the \dword{apa} group.
Work on cabling during this assembly process is performed by \dword{pd}S and \dword{ce} groups under supervision of the \dword{apa} group.
Once the \dwords{apa} will be moved inside the cryostat, the \dword{pd}S and \dword{ce} consortia will be responsible for the routing of the cables in the trays hanging from the top of the cryostat. 

Coldbox testing of integrated  \dword{apa}-\dword{ce}-\dword{pd} units will be done in underground prior to installation into the cryostat. To mitigate the imnplied risks, \dword{pd} modules will be criogenically tested at CSU prior shipping to ITF.

%%%%%%%%%%%%%%%%%%%%%%%%%%%%%%%%%%%
\subsection{Installation into the Cryostat and Cabling}
\label{sec:fdsp-pd-install-pd-cryo}

%The \dword{pd} modules are installed into the \dwords{apa}. There are ten \dword{pd}'s per \dword{apa}, inserted into alternating sides of the \dword{apa} frame, five from each direction. Once a \dword{pd} is inserted, it is attached mechanically to the \dword{apa} frame  and cabled up with a single power/readout cable. Following \dword{pd} installation cold electronics (\dword{ce}) units are installed at the top of the \dword{apa} frame. Cable connection will occur automatically with the modules integration. 

After the \dword{apa} has been integrated with the \dword{pd}S and \dword{ce}, it will be moved to underground 
%via the rails in the clean room to the integrated 
cold test stand 
%for testing 
and then be moved into the cryostat. The two anode planes of the TPC will be assembled inside the cryostat, each of the fully tested \dwords{apa} mechanically linked together. Signal cables from the TPC readout and the \dword{pd} modules are routed up to the feedthrough flanges on the cryostat top side. The cables from each of the \dword{ce} and \dword{pd}'s on the \dword{apa} are then routed and connected to the final flanges on the cryostat.  These assembly, testing and final cabling steps will occur as part of APA ind CE installation (PD cables from the APAs to the flange will be connected as part of the CE cabling installation).  Specialized PD personnel will be provided as part of the installation effort, but the entire installation effort will be planned and directed by the integration and installation group.

%%%%%%%%%%%%%%%%%%%%%%%%%%%%%%%%%%%
\subsection{Calibration and Monitoring}
\label{sec:fdsp-pd-install-calib}
%\todo{Content: Djurcic}

\fixme{Just the integration issues in this section.}

Commissioning of the \dword{spmod} \dword{pds} will rely heavily on the readout electronics, \dword{daq}, and calibration and monitoring system.  Deployment and testing of the readout electronics separately from the in situ installation of photon detector modules in the \dword{apa} is important to establish their proper functioning before connection to the photon detectors or their flanges.  Careful checking at each step of the integration process will help to find unexpected problems early enough to be corrected before individual units are mounted into the larger systems (first in the \dword{apa} then after installation in the cryostat). 

Once the electronics are read out out via the \dword{daq} system, it will be appropriate to add the \dword{pd} modules and continue commissioning of the installed system.  In order to be properly tested the \dword{pd} modules will have to be in the dark.  Making sure it is possible to make this check frequently enough to catch problems early is critical. This will have to be balanced with the needs of installation, as work progresses.  

Once the basic operation of the readout system is established, the calibration and monitoring system will be of great use during the commissioning.  While the background signals from the warm photon detectors may make calibration difficult, the monitoring system will be able to flash UV light to excite the \dword{pd} modules.  These light signals can be used to determine that cabling is connected, and connected properly by looking at light from different UV emitters.  Once the detector is beginning to cool down, the operation of the calibration and monitoring system will become even more important as the monitoring of the individual channels should be a good indication of their proper operation, and again, the proper cabling and interface.

%>> Revision: Ernesto Kemp, Yasas Onel & David Warner Nov/23/2018 <<<<<<<<<<<<<

 

