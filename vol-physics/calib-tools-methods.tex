%\chapter{Tools and Methods}
\chapter{In the tools and methods chapter in Physics}
% Section (4.3 OF PHYSICS VOLUME) 
%%%%%%%%%%%%%%%%%%%%%%%%%%%%%%%%%%%%%%%%%%%%%%%%%%%%%%%
\section{Tools and Methods Employed for Calibration }
\label{sec:phys-tools-calib}

%\fixme{(from SG email 1 Oct about the outline) "We also think a section called calibration tools and methods in the Tools and Methods Employed chapter is useful so we can list calibration related tools and methods here. We envision this chapter including dedicated tools that we are currently developing for alignment related effects, E-field effects and for LBL propagation. It is possible that we may be able to absorb this in calibration detector chapters but we would like to keep this as a placeholder for now."
%Depending on what went into the other parts of the T and M section, we should include the following: 1. Alignment related tools (e.g. hit shifter module) -- SG contacted Tom J. for some text to start on this. 2. E-field distortion propagation tools (e.g. Mooney et al) -- contact Mooney for some text? 3. Noise model should be discussed, not currently anywhere. Can revisit after Dec 18 2018 for CTF+CE. 3. Propagation of detector effects to LBL sensitivities  is discussed in 5.8 4. what else?}

This section describes the tools and methods employed for propagating various detector effects in simulation and understanding the impact of calibrations on physics. 

\textbf{Propagation of detector effects to LBL sensitivities:} \fixme{\textbf{Placeholder:} Coordinate with LBL as this will be covered in Section 5.8 of chapter 5 physics volume; currently don't see it. Largely need to summarize the method used in Elizabeth's talk at the CTF meeting: https://indico.fnal.gov/event/18688/contribution/2/material/slides/0.pdf}

\textbf{Misalignment:} The impact of misalignment on physics can be studied in two steps: 1. a parameterization is needed to estimate the impact of misalignment on the physics. This can be done using what is called a charge drift distortion service that the simulation module can use to produce distorted simulation. The plan is to reconstruct distorted muons with nominal reconstruction and see if it distorts the multiple-coloumb scattering (MCS) momentum measurement. 2. The second step is what the end goal is for the operating far detector. We would like to constrain
all of the APA (and CPA) locations with cosmic rays and apply these as alignment constants to the
reconstruction step so that downstream algorithms are minimally impacted by misalignment. The reason for doing the first step is to estimate the impact on the physics of residual uncorrected misalignment once the second step is done.

\textbf{Propagating E field non-uniformity:} 

Maps of electric field distortions are produced from both ionization (e.g. cosmic rays, ${}^{39}$Ar) and non-ionization sources (e.g. drift field deformations) and are propagated in simulation with appropriate knobs to turn on\slash off the spatial and recombination effects. The former arises as E field distorts the reconstructed position of ionization electron clusters detected by the TPC wire planes and the latter arises due to the dependence of electron-ion recombination on E field.
\fixme{\textbf{placeholder text:} Mooney  et al., to provide more refined text.}

\textbf{Noise Model:} \fixme{\textbf{placeholder:} Noise model should be discussed, not currently anywhere. Check with editors what is the plan. %Can revisit after Dec 18 2018 for CTF+CE.
}



