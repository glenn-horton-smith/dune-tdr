%%%%%%%%%%%%%%%%%%%%%%%%%%%%%%%%%%%%%%%%%%%%%%%%%%%%%%%%%%%%%%%%
\section{Oscillation Physics}
\label{sec:landscape-osc}


Sample things:

\fixme{This is a fixme}

This is a dunefigure - please use for all figures. Figures go in graphics folder. "graphics/" should not go in path.

\begin{dunefigure}[optional caption for LoF]{fig:required-label}
{required full caption (Credit: xyz)}
\includegraphics[width=0.8\textwidth]{dunelogo_colorhoriz}
\end{dunefigure}

\subsection{Other Stuff}
\label{sec:landscape-addl-stuff}

This is a dunetable in a subsection. You can also do subsubsections.

\begin{dunetable}
[The LoT caption]
{cc}
{tab:table-label}
{The full caption that appears above the table.}
Rows & Counts \\ \toprowrule
Row 1 & First \\ \colhline
Row 2 & Second \\ \colhline
Row 3 & Third \\
\end{dunetable}

This is how to use terms from the glossary: It is important to have \dword{dqm} in an experiment. \dword{dune} needs a \dword{daq} system.  Look at common/glossary.tex to see what's there! If you have something you need to add, please notify Luke Corwin.

Here's how to do a reference~\cite{Beacom:2010kk}. If you need to add any references to common/tdr-citedb.bib, please notify Luke. 

This is probably the bulk of what you need to know!


%%%%%%%%%%%%%%%%%%%%%%%%%%%%%%%
\subsection{Three-flavor oscillation phenomenology}
\label{sec:landscape-osc-3flavor}

%%%%%%%%%%%%%%%%%%%%%%%%%%%%%%%
\subsection{Connections with leptogenesis} 
\label{sec:landscape-osc-leptogen}


%%%%%%%%%%%%%%%%%%%%%%%%%%%%%%%
\subsection{Impacts of currently running experiments}
\label{sec:landscape-osc-impacts}
